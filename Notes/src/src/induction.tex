\section{Induction}

Induciton is an important tool in computer science, as it provides us one with
an procedure to show correctness of our algorithms, and two can be directly
translated to recursion.

\subsection{Definition and Examples}

\begin{definition*}

\textbf{Induction} is a mathematical proof technique. Formally, suppose we have
some program $P$, which has input $n$. Furthermore, suppose we can show that:
\begin{enumerate}[1)]
\item $[$Base Case$]$ $P(0)$ is provably correct.
\item $[$Inductive Step$]$ Suppose that $P(0), \dots, P(k)$ is correct for some
fixed $k \in \N$.  Then we can show that $P(k+1)$ is provably correct.
\end{enumerate}
Then it must be true that $\forall n, P(n)$ is correct. (Note that we don't need
for \textit{zero} that $P(0)$ is provably correct, but rather that $P(n_0)$ is
provably correct, for some fixed $n_0$. We will typically take $n_0 = 0, 1$.)

Conceptually you can envision this as a staircase. Suppose that you can stand at
the bottom of the staircase, and that if you're on step $k$, you can move to
step $k+1$. Then we know that $\forall n$, we can stand on step $n$.

\end{definition*}

Take the following as applications of induction.

\begin{itemize}

\item Prove that $1 + 2 + \dots + n = \frac{n(n+1)}{2}$.

\begin{proof}

Clearly for $n = 1$, we have that $1 = \frac{1(1+1)}{2} = 1$. Now suppose that
$1 + 2 + \dots + n = \frac{n(n+1)}{2}$. We want to show that $1 + \dots + n+1 =
\frac{(n+1)(n+2)}{2}$. Notice that:
$$
\sum_{i=1}^{n+1} i = n+1 + \sum_{i=1}^{n}i = n+1 + \frac{n(n+1)}{2}
$$
The last equality being true by our assumption. Now notice we can write
$n(n+1)/2 = (n^2 + n)/2$, and we can write $n+1 = \frac{2n + 2}{2}$. Therefore
adding the two expresions together results in $(n^2 + 3n + 2)/2 =
(n+1)(n+2)/2$.

\end{proof}

\item We call a \textbf{perfect binary tree} a tree structure such that each
non-leaf vertex has exactly two children. So for example, a perfect binary tree
of height $2$ would be:

\Tree [.a [.b d e ] [.c f g ] ]

Prove that a perfect binary tree of height $n$ has exactly $2^n - 1$ vertices.

\begin{proof}

First suppose $n = 1$. Clearly a tree of height $1$ has only one vertex, and
indeed $2^1 - 1 = 1$.

Now suppose a tree of height $n$ has $2^n - 1$ vertices. We want to show that a
tree of height $n+1$ must have $2^{n+1} - 1$ vertices. Consider the root of such
a tree (in the example above it's vertex $a$). Notice that it's left subchild is
a perfect binary tree of height $n$, and so is it's right subtree. Therefore,
counting the verteces in each of those subtrees using our supposition that a
tree of height $n$ has $2^n - 1$ vertices gives us a count of $2(2^n - 1)$ for
the vertices under the root. Therefore a tree of height $n+1$ has $2(2^n - 1) +
1$ vertices ($+1$ for the root). Notice that:
$$
2(2^n - 1) + 1 = 2^{n+1} -2  + 1 = 2^{n+1} - 1
$$
and therefore we've shown our inductive step.

Thus, a perfect binary tree of height $n$ has $2^n - 1$ vertices.

\end{proof}

\end{itemize}

\subsection{Towers of Hanoi and thinking recursively}

The reason why we're reviewing induction, is that induction is very (read:
\textit{\textbf{VERY}}) useful for proving correctness for recursive functions.
You will use it in literally every single part of the rest of the course so it
pays to get it down well. A classical example is Towers of Hanoi.

\begin{definition*}

Towers of Hanoi is a mathematical puzzle. Suppose we have three rods, and a
number of disks of different sizes $1 \to n$. which can slide onto any rod. The
puzzle starts with the disks all arranged on the first rod in descending
order such that $n$ is at the bottom, and $1$ is at the top. For example, for $n
= 3$, the beginning configuration is:

$$
\begin{matrix}
-   & | & | \\
--  & | & | \\ 
--- & | & | \\ 
A & B & C 
\end{matrix}
$$

The goal of this game is to move the entire stack of rings from pole $A$ to pole
$B$, while obeying the constraints:

\begin{itemize}

\item You may move one disk at a time with the command $Move(arg1, arg2)$ where
this moves the disk on top of pole $arg1$ to the top of pole $arg2$.

\item No larger disk may be placed on top of a smaller disk.

\end{itemize}

\end{definition*}

We want to write an algorithm to solve this puzzle, but to also do so with a
minimal amount of $Move$ calls.


